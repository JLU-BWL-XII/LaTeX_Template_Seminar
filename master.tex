% ------------------------------------------------------------------------
% Template zur Erstellung von Seminararbeiten an der Professur für
% Digitalisierung, E-Business und Operations Management
% Erstellerin: Jella Pfeiffer
% Datum 08.10.2019
% Überarbeitet von Pascal Heßler am 19.08.2020
% ------------------------------------------------------------------------

% ------------------------------------------------------------------------
% Allgemeine Einstellungen
% ------------------------------------------------------------------------
\documentclass[12pt]{article}

\usepackage[paper=a4paper,left=30mm,right=30mm,top=40mm,bottom=40mm]{geometry}
\usepackage[utf8]{inputenc}       % Damit können Umlaute ganz normal geschrieben werden.
% \usepackage[english]{babel}       % Verwende deutsche, bzw. amerikanische Silbentrennung
\usepackage[ngerman]{babel}       % Deutsche Alternative
\usepackage{setspace}             % Zeilenabstand imlpementiert den Befehl 
% ------------------------------------------------------------------------
% Literatur
% ------------------------------------------------------------------------
% Achtung zur Einbindung der Literatur wird hier Biber verwendet und nich der Standard BibTex. In Sublime bracu man nichts verändern.
% 1. Sicherstellen, dass Biber installiert ist ->  in Tex Live vorinstalliert
% 2. Stellt eure Schreibumgebung um -> In TeXstudio: Optionen -> TeXstudio konfigurieren -> Erzeugen -> Standard Bibliographieprogramm
\usepackage[backend=biber,style=apa,natbib=false]{biblatex}
\DeclareLanguageMapping{american}{american-apa}         % Sprach Einstellung

\addbibresource{BibliographieAbschlussarbeit.bib}       % Legt die Datei fest in der die Literatur liegt (z. B.: Export aus Citavi)

\usepackage{breakcites}                                 % Falls Zitationen nicht ans Zeilenende passen
\usepackage{csquotes}                                   % Erweitert die Möglichkeiten beim zitieren
% ------------------------------------------------------------------------
% Mathematische Symbole
% ------------------------------------------------------------------------
\usepackage{amssymb}
\usepackage{amsmath}        % Formattierung von Tabellen und Matritzen
\usepackage{amsfonts}

% ------------------------------------------------------------------------
% Grafiken
% ------------------------------------------------------------------------
\usepackage{graphicx}       % Zum Einbinden von Grafiken
\usepackage{subfigure}      % Um Bilder innerhalb einer Figure anzuordnen (Bild a b c d...)

% ------------------------------------------------------------------------
% Tabellen
% Um einfache Tabellen zu erstellen kann https://www.tablesgenerator.com/ verwendet werden.
% ------------------------------------------------------------------------
\usepackage{tabularx}       % Ermöglicht weitere Tabellen Einstellungen
\usepackage{multirow}       % Ermöglicht Zellen zu verbinden
\usepackage{booktabs}       % Tabellen Midline Topline etc.
\usepackage{float}          % Tabelen Positionierung
\usepackage{longtable}      % Ermöglicht Tabellen über mehrere Seiten hinweg, so dass sie noch dasselbe Format besitzten ftp://ftp.fu-berlin.de/tex/CTAN/macros/latex/required/tools/longtable.pdf
% **********************************************************************************************************************
% Folgende Pakete sind rein optional und etwas komplexer in ihrer Umsetzung nur für etwas erfahrenere LaTeX User
% **********************************************************************************************************************
% \usepackage{siunitx}      % Ausrichten von Tabellen an dezimalstellen (Umsetzung etwas komplizierter)
% \usepackage{rccol}        % Ein Neues Format R (wird wahrscheinlich mit dem Folgenden R zu Problemen führen) und sorgt, dass nach , ausgerichtet wird und automatisches runden in Tabellen wird ermöglicht
% \usepackage{fltpoint}     % Gehört zu rccol
% \usepackage{dcolumn}      % Gehört zu rccol
%
% \newcolumntype{L}[1]{>{\raggedright\arraybackslash}p{#1}}       % Linksbündig mit Breitenangabe
% \newcolumntype{C}[1]{>{\centering\arraybackslash}p{#1}}         % Zentriert mit Breitenangabe
% \newcolumntype{R}[1]{>{\raggedleft\arraybackslash}p{#1}}        % Rechtsbündig mit Breitenangabe
% \newcolumntype{x}[1]{!{\centering\arraybackslash\vrule width #1}}

% ------------------------------------------------------------------------
% Sonstige
% ------------------------------------------------------------------------
\usepackage{scrhack}        % Löst unnötige Warnungen!
\usepackage{color}          % Ermöglicht Text einzufärben \pagecolor{FARBE}  \color{FARBE} \textcolor{FARBE}{TEXT} \colorbox{FARBE}{TEXT}
\usepackage{hyperref}       % Ermöglicht URLS schreibweise ist z.B. \url{http://www.uni-giessen.de}

\usepackage{eurosym}        % Für Euro Symbol \euro{} oder \eruo (Unterschieden sich bezüglich Leerzeichen danach)

\usepackage{framed}         % Einrahmen von Texten mit der shaded-Umgebung.

\usepackage{pdflscape}      % Querformat möglich über \begin{landscape} \end{landscape}
% ------------------------------------------------------------------------
% Fein Justierungen
% ------------------------------------------------------------------------
\DeclareMathOperator*{\argmax}{arg\,max}

\newenvironment{packed_enum}{
\begin{enumerate}
  \setlength{\itemsep}{1pt}
  \setlength{\parskip}{0pt}
  \setlength{\parsep}{0pt}
}{\end{enumerate}}

\hypersetup{hidelinks}                 % Versteckt die Links im PDF (optisch)
% ----- ende der präambel ----------------------------------
% Start des Dokuments
% ----------------------------------------------------------
\begin{document} 
% Theoretisch kann man auch alles in einem Dokument haben. Das ist jedoch unübersichtlich. Daher werden hier die einzelnen Kapitel eingebunden.
\pagenumbering{Alph}                  % Löst unnötigen Warnungen
% Die Titelseite der Arbeit

\begin{titlepage}
\begin{center} % zentrieren

  % Logo der Uni Giessen bitte suchen und einfuegen
  \begin{figure}[ht]
    \centering
    \includegraphics{graphics/logo.png}
  \end{figure}

  % Vertikaler Zwischenraum
  \bigskip
  \vfill
  % Titel der Arbeit und Typ der Arbeit, umrandet
    \begin{framed}
    \begin{center}
      \textsc{{\Large Titel\\
      Untertitel\\}}
                                % Letztes \\ ist wichtig, beginnt eine neue Zeile f{\"u}r die Art der Arbeit

      \bigskip

                                % Art der Arbeit, ggf. auszutauschen gegen Seminar- oder Bachelorarbeit
      \textbf{Seminararbeit}
    \end{center}
    \end{framed}
    \vfill
    \vfill


  % Daten des Erstellers, Einreichungsdatum
  % in einer Tabelle ausgerichtet
  \begin{tabular*}{0.62\textwidth}{r@{\extracolsep{\fill}}l}
    eingereicht im: & Oktober 2019\\\\
    von: & XXX\\
    & geboren am XX. Oktober 2010\\
    \\
    Matrikelnummer: & XXX\\
    Studiengang: & XXX\\
    Private Adresse: & XXX\\
    Telefonnummer: & XXX\\
    E-Mail-Adresse: & XXX
  \end{tabular*}
  \vfill
  \vfill


  \rule{\textwidth}{.4pt}\\ % vertikale Linie
  Justus-Liebig-Universität Gießen\\
  Professur f\"ur Digitalisierung, E-Business und Operations Management\\
  35394  Gie\ss{}en\\
  \small \url{https://www.uni-giessen.de/fbz/fb02/fb/professuren/bwl/e-business-operations-management}
\end{center}

\end{titlepage} % Ende des Titelblatts

%%% Local Variables:
%%% mode: latex
%%% TeX-master: "~/Documents/DA-Vorlage/beispiel/da-beispiel"
%%% End:
              % Titelseite einbinden
\pagenumbering{arabic}                % Löst unnötigen Warnungen
\include{sections/abstract}
\section{Introduction} \label{sec:introduction}

Wissenschaftliche Arbeiten weisen standardisierte Strukturen auf, von denen man im Aus- nahmefall abweichen kann. Eine Standardstruktur für empirische Arbeiten ist die folgende:
\begin{itemize}
  \item Abstract
  \item Introduction [Einleitung]
  \item Related Work [Grundlagen und Verwandte Arbeiten]
  \item Method/Experimental Design/Implementation
  \item Results
  \item Discussion
  \item Conclusions
  \begin{itemize}
    \item Summary (ist optional!)
    \item Limitations and Future Research [Limitationen und Ausblick]
     \item Contribution [Fazit/Schlussfolgerung]
   \end{itemize}
   \item References [Literaturverzeichnis]
   \item Appendix [Anhang]
\end{itemize}


Literaturbasierte Arbeiten verfolgen weniger eine Standardstruktur. Orientieren können Sie sich heran:
\begin{itemize}
  \item	Abstract
  \item Introduction [Einleitung]
  \item Related Work [Grundlagen und Verwandte Arbeiten]: Es kann sein, dass Sie dieses Kapitel besser über mehrere Kapitel mit unterschiedlichen Schwerpunkten verteilen und diese dann auch spezifisch in der Kapitelüberschrift entsprechend des Inhalts benennen (und nicht einfach related work).
  \item Method/Experimental Design/Implementation: Dieses Kapitel sollten Sie weglassen, falls Sie nicht viel zu Ihrer Methodik sagen können. Aber selbst wenn Sie nur eine literaturbasierte Arbeit und keine empirische Schreiben, können Sie hier sehr strukturiert vorgehen und dann hier Ihre Literaturarbeit beschreiben, z.B. nach welchen Keywords haben Sie gesucht und wo, welche gefundene Quellen haben Sie ein/ausgeschlossen, etc..
  \item Results: Dieses Kapitel heißt oft anders in rein literaturbasierten Arbeiten und hat dann einen für die Arbeit spezifischen Titel, wie z.B. Risiko-Nutzen-Analyse der Einführung von Industrie 4.0. Hier ist der Teil, der größten Eigenleistung, in der Sie selbst interpretieren, Frameworks formulieren, Modelle formulieren oder ähnliches
       \item Conclusions
  \begin{itemize}
    \item Summary (ist optional!)
    \item Limitations and Future Research [Limitationen und Ausblick]
     \item Contribution [Fazit/Schlussfolgerung]
   \end{itemize}
  \item References [Literaturverzeichnis]
  \item Appendix [Anhang]
\end{itemize}

Die typische Struktur der Einleitung ist:
\begin{enumerate}
  \item Problemstellung und Motivation
  \item Stand der Forschung, darauf aufbauend Forschungslücke und Forschungsfrage(n) herausarbeiten
  \item Ziel der Arbeit und eigener methodischer Ansatz zur Beantwortung der Forschungsfrage(n): Absatz beginnt meist mit: The goal of this thesis/work/manuscript is
  \item Ergebnisse der Arbeit [optional]
  \item Erwarteter wissenschaftlicher (und praktischer) Beitrag [=Contribution]
  \item Manchmal folgt noch eine Gliederung [ausformuliert]
\end{enumerate}

%%% Local Variables:
%%% mode: latex
%%% TeX-master: "..\\da-beispiel"
%%% End:

\include{sections/figurestables}
\include{sections/combinatorialAuctions}
\section{Conclusions}\label{sec:conclusions}

Auf Deutsch: Fazit

\subsection{Limitations and Future Research}

Auf Deutsch: Limitationen und Ausblick

Es ist sinnvoll, jede Limitation an eine Idee zu knüpfen, wie diese in zukünftigen Arbeit zu adressieren waere.

\subsection{Contribution}

Auf Deutsch: Wissenschaftlicher und Praktischer Beitrag

Was sind die Beitraege Ihrer Arbeit sowohl für die Wissenschaft (und Theorie) als auch für die Praxis? Hier sollten Sie versuchen über den Tellerrand hinauszuschauen und einen eher weiten Blick einnehmen.






% ------------------------------------------------------------------------
% Erstellen des Literaturverzeichnisses
% ------------------------------------------------------------------------
\printbibliography[heading=bibintoc]  
\include{sections/appendix}
\include{sections/erklaerung}


\end{document}
% ----------------------------------------------------------
% Ende
% ----------------------------------------------------------
