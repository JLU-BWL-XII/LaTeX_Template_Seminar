\section{Introduction} \label{sec:introduction}

Wissenschaftliche Arbeiten weisen standardisierte Strukturen auf, von denen man im Aus- nahmefall abweichen kann. Eine Standardstruktur für empirische Arbeiten ist die folgende:
\begin{itemize}
  \item Abstract
  \item Introduction [Einleitung]
  \item Related Work [Grundlagen und Verwandte Arbeiten]
  \item Method/Experimental Design/Implementation
  \item Results
  \item Discussion
  \item Conclusions
  \begin{itemize}
    \item Summary (ist optional!)
    \item Limitations and Future Research [Limitationen und Ausblick]
     \item Contribution [Fazit/Schlussfolgerung]
   \end{itemize}
   \item References [Literaturverzeichnis]
   \item Appendix [Anhang]
\end{itemize}


Literaturbasierte Arbeiten verfolgen weniger eine Standardstruktur. Orientieren können Sie sich heran:
\begin{itemize}
  \item	Abstract
  \item Introduction [Einleitung]
  \item Related Work [Grundlagen und Verwandte Arbeiten]: Es kann sein, dass Sie dieses Kapitel besser über mehrere Kapitel mit unterschiedlichen Schwerpunkten verteilen und diese dann auch spezifisch in der Kapitelüberschrift entsprechend des Inhalts benennen (und nicht einfach related work).
  \item Method/Experimental Design/Implementation: Dieses Kapitel sollten Sie weglassen, falls Sie nicht viel zu Ihrer Methodik sagen können. Aber selbst wenn Sie nur eine literaturbasierte Arbeit und keine empirische Schreiben, können Sie hier sehr strukturiert vorgehen und dann hier Ihre Literaturarbeit beschreiben, z.B. nach welchen Keywords haben Sie gesucht und wo, welche gefundene Quellen haben Sie ein/ausgeschlossen, etc..
  \item Results: Dieses Kapitel heißt oft anders in rein literaturbasierten Arbeiten und hat dann einen für die Arbeit spezifischen Titel, wie z.B. Risiko-Nutzen-Analyse der Einführung von Industrie 4.0. Hier ist der Teil, der größten Eigenleistung, in der Sie selbst interpretieren, Frameworks formulieren, Modelle formulieren oder ähnliches
       \item Conclusions
  \begin{itemize}
    \item Summary (ist optional!)
    \item Limitations and Future Research [Limitationen und Ausblick]
     \item Contribution [Fazit/Schlussfolgerung]
   \end{itemize}
  \item References [Literaturverzeichnis]
  \item Appendix [Anhang]
\end{itemize}

Die typische Struktur der Einleitung ist:
\begin{enumerate}
  \item Problemstellung und Motivation
  \item Stand der Forschung, darauf aufbauend Forschungslücke und Forschungsfrage(n) herausarbeiten
  \item Ziel der Arbeit und eigener methodischer Ansatz zur Beantwortung der Forschungsfrage(n): Absatz beginnt meist mit: The goal of this thesis/work/manuscript is
  \item Ergebnisse der Arbeit [optional]
  \item Erwarteter wissenschaftlicher (und praktischer) Beitrag [=Contribution]
  \item Manchmal folgt noch eine Gliederung [ausformuliert]
\end{enumerate}

%%% Local Variables:
%%% mode: latex
%%% TeX-master: "..\\da-beispiel"
%%% End:
