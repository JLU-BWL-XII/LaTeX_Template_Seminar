% Die Titelseite der Arbeit

\begin{titlepage}
\begin{center} % zentrieren

  % Logo der Uni Giessen bitte suchen und einfuegen
  \begin{figure}[ht]
    \centering
    \includegraphics{graphics/logo.png}
  \end{figure}

  % Vertikaler Zwischenraum
  \bigskip
  \vfill
  % Titel der Arbeit und Typ der Arbeit, umrandet
    \begin{framed}
    \begin{center}
      \textsc{{\Large Titel\\
      Untertitel\\}}
                                % Letztes \\ ist wichtig, beginnt eine neue Zeile f{\"u}r die Art der Arbeit

      \bigskip

                                % Art der Arbeit, ggf. auszutauschen gegen Seminar- oder Bachelorarbeit
      \textbf{Seminararbeit}
    \end{center}
    \end{framed}
    \vfill
    \vfill


  % Daten des Erstellers, Einreichungsdatum
  % in einer Tabelle ausgerichtet
  \begin{tabular*}{0.62\textwidth}{r@{\extracolsep{\fill}}l}
    eingereicht im: & Oktober 2019\\\\
    von: & XXX\\
    & geboren am XX. Oktober 2010\\
    \\
    Matrikelnummer: & XXX\\
    Studiengang: & XXX\\
    Private Adresse: & XXX\\
    Telefonnummer: & XXX\\
    E-Mail-Adresse: & XXX
  \end{tabular*}
  \vfill
  \vfill


  \rule{\textwidth}{.4pt}\\ % vertikale Linie
  Justus-Liebig-Universität Gießen\\
  Professur f\"ur Digitalisierung, E-Business und Operations Management\\
  35394  Gie\ss{}en\\
  \small \url{https://www.uni-giessen.de/fbz/fb02/fb/professuren/bwl/e-business-operations-management}
\end{center}

\end{titlepage} % Ende des Titelblatts

%%% Local Variables:
%%% mode: latex
%%% TeX-master: "~/Documents/DA-Vorlage/beispiel/da-beispiel"
%%% End:
